
  \section{Auswertung}
  \subsection{Messergebnisse}
  \begin{table}[h]
    \centering
    \caption{Messdaten für Temperatur und Druck}
    \label{tab:a}
    \begin{tabular}{S S S S}
      \toprule
      \multicolumn{2}{c}{Druck < 1 bar} & \multicolumn{2}{c}{Druck > 1 bar} \\
      {$T/\si{\celsius}$} & {$p/\si{\milli\bar}$} & {$T/\si{\celsius}$} & {$p/\si{\bar}$} \\
      \midrule
      45 & 147 & 103 & 1.13 \\
      50 & 164 & 106 & 1.27 \\
      55 & 186 & 109 & 1.43 \\
      60 & 212 & 112 & 1.60 \\
      65 & 250 & 115 & 1.79 \\
      70 & 311 & 118 & 2.00 \\
      75 & 383 & 121 & 2.22 \\
      80 & 475 & 124 & 2.47 \\
      85 & 582 & 127 & 2.74 \\
      90 & 698 &  &  \\
      95 & 852 &  &  \\
      100 & 1011 &  &  \\
      \bottomrule
    \end{tabular}
  \end{table}
  Auf der linken Tabellenhälfte \ref{tab:a} sind die Messwerte für Druck
  ($p < 1$\si{bar}) und Temperatur aufgezeichnet,
  die mit Apparatur 1(siehe Abbildung \ref{fig: Apparatur1}) aufgenommen wurden. Auf der rechten Hälfte sind die
  Messwerte, die mit Apparatur 2(siehe Abbildung \ref{fig: Apparatur2}) aufgenommen wurden,
  allerdings wurde hier der Offset-Druck bereits mit eingerechnet. \\ \\
  Offsetdruck für die zweite Messung ($p > 1\si{bar}$):
  \begin{equation}
    p_\text{offset} = 0.2 bar
  \end{equation}

  \subsection{Graphen und Rechnung}
  \subsubsection{Messung mit Apparatur 1}
  In dem Graphen \ref{fig:p1(T)} ($p < 1$\si{bar})
  ist der Logarithmus des Dampfdruckes $p$ (in \si{\kPa})
  gegen die reziproke absolute Temperatur (in \si{\kelvin}) dargestellt.
  Mit der Funktion linregress aus scipy.stats
  wurde eine Ausgleichsgerade erstellt, für die folgende Werte für den
  Y-Achsen-Abschnitt $b$ und die Steigung $m$ zurückgegeben wurden (umgerechnet
  von \si{\kPa} in \si{\Pa}):
  \begin{equation}
    p_1(T) = m \cdot T + b
  \end{equation}
  \begin{align}
    b & = 29.998 &  m & = \num{-4344.8 +- 169.3}
  \end{align}
  \begin{figure}
    \centering
    \includegraphics[height = 8.30 cm]{plot.pdf}
    \caption{Messwerte von Apparatur 1, Graph p1(T)}
    \label{fig:p1(T)}
  \end{figure}

  Die gemittelte Verdampfungswärme $L$ pro \si{mol} für den Druckbereich
  $p < 1\si{bar}$ lautet nun
  \begin{equation}
    L = -m \cdot R \cdot \si{\kelvin} =
    \SI{36122.6672 +- 1407.5602}{\joule\per\mol}
  \end{equation}
  mit der allgemeinen Gaskonstante\cite{gaskonstante}
  $R = 8.314\si[per-mode=reciprocal]{\joule\per\kelvin\per\mol}$.
  Die äußere Verdampfungswärme $L_a$ pro \si{mol} bei $T = 373\si{\kelvin}$ ist
  \begin{equation}
    L_a = T \cdot R = 3101.122 \si{\joule\per\mol}.
  \end{equation}
  Aus diesen Werten ergibt sich dann die innere Verdampfungswärme $L_i$, die
  zur Überwindung der molekularen Anziehungskräfte bei der Verdampfung benötigt
  wird:
  \begin{equation}
    L_i = L - L_a = \SI{2.0610428289(0878530014)e27}{\electronvolt\per\mol}
  \end{equation}
  Um die innere Verdampfungswärme pro Molekül zu bestimmten muss man noch durch
  die Avogadro-konstante\cite{avogadro}
  ($N_A = 6.022 \cdot 10^{23} \si[per-mode=reciprocal]{\per\mol}$) teilen:
  \begin{equation}
    L_{i,m} = \frac{L_i}{N_A} = \SI{3422.5221336 +- 145.8867509}{\electronvolt}
  \end{equation}
  \newpage
   \subsubsection{Messung mit Apparatur 2}
  \label{sec: MmA2}
  In dem Graphen \ref{fig:p2(T)} ($p > 1$\si{\bar}) ist der Dampfdruck
  $p$ (in \si{\kPa}) gegenüber der absoluten Temperatur $T$ (in \si{\kelvin})
  aufgetragen. Das Ausgleichspolynom dritten Grades wurde mit Hilfe der
  Funktion polyfit aus numpy erstellt.
  Dabei wurden folgende Werte für die Funktionsgleichung zurückgegeben
  (umgerechnet von \si{\kPa} in \si{\Pa}):
  \begin{equation}
    p_2(T) = aT³ + bT² + cT + d
  \end{equation}
  \begin{align}
    a & = 0.84 & b & = -880.01 \\
    c & = 309277.95 & d & = -36508852.61
  \end{align}
  \begin{figure}[h]
    \centering
    \includegraphics[height = 8.30 cm]{plot2.pdf}
    \caption{Messwerte von Apparatur 2, Graph p2(T)}
    \label{fig:p2(T)}
  \end{figure}
  \\
  Um eine Abhängigkeit zwischen der Verdampfungswärme $L$ und der Temperatur
  $T$ bei einem Druck $p > 1\si{\bar}$ herauszufinden löst man zunächst die
  Clausius-Clapeyronsche Gleichung nach $L$ auf.
  \begin{equation}
    (V_D - V_F) dp = \frac{L}{T} dT
    \iff L = (V_D - V_F) \frac{dp}{dT} T
  \end{equation}
  Nun ist der Differentialquotient $\frac{dp}{dT}$ die Ableitung der
  Ausgleichsfunktion $p_2(T)$:
  \begin{equation}
    p_2'(T) = 2.25 T² - 1760.02 T + 309277.95
  \end{equation}
  Das Flüssigkeits-Volumen $V_F$ pro Mol wird über die molare Masse und die
  Dichte von Wasser\cite{stoffwerte}
  (bei $T = 373 K$, $p = 100000 \si{\Pa}$) berechnet:
  \begin{align}
    M_W & = \underbrace{2 \cdot 1}_{H_2} + \underbrace{16}_{O}
    \si{\gram\per\mol} & \rho_W & = 950000 \si{\gram\per\cubic\meter}
  \end{align}
  \begin{equation}
    V_F = \frac{M_W}{\rho_W} = 1.894736 \cdot 10^{-5} \si{\cubic\meter\per\mol}
  \end{equation}
  Das Dampf-Volumen $V_D$ pro Mol kann über die folgende Gleichung genähert
  werden:
  \begin{equation}
    \Bigl(p + \frac{a}{V²}\Bigr)V = R T
  \end{equation} mit $T = 373 \si{\kelvin}$, $p = 100000 \si{\Pa}$ und
  $a = 0.9 \si[per-mode=reciprocal]{\joule\cubic\meter\per\mol\squared}$.
  Durch umformen mit der \\
  pq-Formel erhält man dann einen sinnvollen Wert
  \begin{equation}
    V_D = 0.03072569441 \si{\cubic\meter\per\mol}
  \end{equation} für das Dampfvolumen.
  Zuletzt setzt man die Werte in die umgeformte
  Clausius-Clapeyronsche Gleichung ein und erhält für die Verdampfungswärme in
  Abhängigkeit von der Temperatur folgende Funktion:
  \begin{equation}
    L(T) = 0.691 \cdot T³ + 54.044 \cdot T² + 9499.919 \cdot T
  \end{equation}
  \newpage
  \subsection{Diskussion der Ergebnisse}
  In Abbildung \ref{fig:p1(T)} kann man erkennen, dass die Messwerte sehr nah
  an der durch die Funktion linregress bestimmten Geraden liegen. Der $r$-Wert
  (r = 0.9924) bestätigt die Annahme, dass die Verdampfungswärme von Wasser im
  Druckbereich $p < 1 \si{\bar}$ annähernd konstant ist. Auffällig ist, dass
  die berechnete äußere Verdampfungswärme $L_a$, die zur Volumenausdehnung (und
  damit zur Druckerhöhung im Gefäß) nötig ist, deutlich kleiner als die innere
  Verdampfungswärme $L_i$, die zur Überwindung der molekularen Anziehungskräfte
  benötigt wird, ist ($L_i > 10 \cdot L_a$).
  Das liegt daran, dass $L_i$ die kinetische Energie $E_\text{kin}$ der Moleküle in
  der Flüssigkeit so erhöhen muss, dass der Anteil mit maximaler kinetischer
  Energie (Maxwellsche Geschwindigkeitsverteilung) wächst.
  Die äußere Verdampfungswärme $L_a$ sorgt lediglich dafür, dass die Moleküle
  im Wasserdampf, die deutlich weniger gebunden sind und mehr Platz haben,
  zusammengedrückt werden ($V = const, p > p_0$).

  In Abbildung \ref{fig:p2(T)} erkennt man, dass Druck und Temperatur ab
  $p > 1\si{\bar}$ anders voneinander abhängen und die latente Wärme $L$
  dadurch nicht mehr als konstant angesehen werden kann. Ein Polynom dritten
  Grades kommt den Messwerten von Druck und temperatur sehr nahe.
  Bei der berechneten Verdampfungswärme $L(T)$ muss man beachten, dass in der zur
  Berechnung benutzten Formel Drücke und Volumina als konstant
  angenommen wurden, was nur stark genähert zutrifft. Den Graphen von $L$ sieht
  man in Abbildung \ref{fig:L}.
  \begin{figure}[h]
    \centering
    \includegraphics[height = 8.30 cm]{plot3.pdf}
    \caption{Verdampfungswärme bei $p > 1\si{\bar}$, Graph $L(T)$}
    \label{fig:L}
  \end{figure}

  \newpage
