\section{Auswertung}
\subsection{Messergebnisse}

\begin{table}[h]
  \centering
  \caption{Messdaten für Temperatur und Druck}
  \label{tab:a}
  \begin{tabular}{S S S S}
    \toprule
    \multicolumn{2}{c}{Druck < 1 bar} & \multicolumn{2}{c}{Druck > 1 bar} \\
    {$T/\si{\celsius}$} & {$P/\si{\milli\bar}$} & {$T/\si{\celsius}$} &
    {$p/\si{\bar}$} \\
    \midrule
    45 & 147 & 103 & 1.13 \\
    50 & 164 & 106 & 1.27 \\
    55 & 186 & 109 & 1.43 \\
    60 & 212 & 112 & 1.60 \\
    65 & 250 & 115 & 1.79 \\
    70 & 311 & 118 & 2.00 \\
    75 & 383 & 121 & 2.22 \\
    80 & 475 & 124 & 2.47 \\
    85 & 582 & 127 & 2.74 \\
    90 & 698 &  &  \\
    95 & 852 &  &  \\
    100 & 1011 &  &  \\
    \bottomrule
  \end{tabular}
\end{table}

Auf der linken Tabellenhälfte \ref{tab:a} sind die Messwerte für Druck
$p < 1$\si{bar} und Temperatur aufgezeichnet,
die mit der Apparatur in Abbildung \ref{fig: Apparatur1} aufgenommen wurden. Auf der
rechten Hälfte sind die Messwerte,
die mit der Apparatur in Abbildung \ref{fig: Apparatur2} aufgenommen wurden,
allerdings wurde hier der Offset-Druck bereits mit eingerechnet. \\ \\
Der Offsetdruck für die zweite Messung beträgt für $p > 1\si{bar}$
\begin{equation}
  p_\text{offset} = \SI{0.2}{\bar}.
\end{equation}

\subsection{Graphen und Rechnung}
\subsubsection{Messung mit Apparatur 1}
In dem Graphen \ref{fig:f1(T)} mit $p < 1$\si{bar}
ist der Logarithmus des Dampfdruckes $p$ in \si{\kPa}
gegen die reziproke absolute Temperatur in $1/\si{\kelvin}$ dargestellt.
Mit der Funktion $linregress$ aus $scipy.stats$
wurde eine Ausgleichsgerade erstellt, für die folgende Werte für den
Y-Achsen-Abschnitt $b$ und die Steigung $m$ zurückgegeben wurden:
\begin{equation}
  f_1(T) = m \cdot T + b
\end{equation}
\begin{align}
  b & = \SI{29.998}{\Pa} &  m & = \SI{-4344(169)}{\per\kelvin}
\end{align}

\newpage

\begin{figure}
  \centering
  \includegraphics[height = 8.30 cm]{plot.pdf}
  \caption{Messwerte von Apparatur 1, Graph f1(T)}
  \label{fig:f1(T)}
\end{figure}

Die gemittelte Verdampfungswärme $L$ pro \si{mol} für den Druckbereich
$p < 1\si{bar}$ lautet nun
\begin{equation}
  L = -m \cdot R = \SI{36123(1408)}{\joule\per\mol}
\end{equation}
mit der allgemeinen Gaskonstante\cite{gaskonstante}
$R = 8.314 \si[per-mode=reciprocal]{\joule\per\kelvin\per\mol}$.
Die äußere Verdampfungswärme $L_a$ pro \si{mol} bei $T = 373
\si{\kelvin}$ ist

\begin{equation}
  L_a = T \cdot R = \SI{3101}{\joule\per\mol}.
\end{equation}

Aus diesen Werten ergibt sich dann die innere Verdampfungswärme $L_i$, die
zur Überwindung der molekularen Anziehungskräfte bei der Verdampfung benötigt
wird:

\begin{equation}
  L_i = L - L_a = \SI{33022(1408)}{\joule\per\mol} =
  \SI{2.06(9)e23}{\electronvolt\per\mol}
\end{equation}

Um die innere Verdampfungswärme pro Molekül zu bestimmten muss man noch durch
die Avogadro-konstante\cite{avogadro}
$N_A = 6.022 \cdot 10^{23} \si[per-mode=reciprocal]{\per\mol}$ teilen:
\begin{equation}
  L_{i,m} = \frac{L_i}{N_A} = \SI{0.34(1)}{\electronvolt}
\end{equation}

\newpage

 \subsubsection{Messung mit Apparatur 2}
\label{sec: MmA2}
In dem Graphen von Abbildung \ref{fig:f2(T)} mit $p > 1$\si{\bar} ist der
Dampfdruck
$p$ in \si{\kPa} gegenüber der absoluten Temperatur $T$ in $1/\si{\kelvin}$
aufgetragen. Mit Hilfe der Funktion $polyfit$ aus $numpy$ wurde ein
Ausgleichspolynom erstellt. Der Grund für die Wahl eines
Polynoms mit dem Grad zwei anstatt wie in der Aufgabe gefordert mit dem Grad
drei oder höher wird in der Diskussion erklärt.
Folgende Werte wurden für die Funktionsgleichung zurückgegeben:
\begin{equation}
  f_2(T) = aT^2 + bT + c
\end{equation}
\begin{align}
  a & = \SI{100.3}{\per\kelvin\tothe{2}} & b & =
  \SI{-71186.3}{\per\kelvin} & c & = 12700.1 \\
\end{align}

\begin{figure}[h]
  \centering
  \includegraphics[height = 8.30 cm]{plot2.pdf}
  \caption{Messwerte von Apparatur 2, Graph f2(T)}
  \label{fig:f2(T)}
\end{figure}

\\
Um eine Abhängigkeit zwischen der Verdampfungswärme $L$ und der Temperatur
$T$ bei einem Druck $p > 1\si{\bar}$ herauszufinden löst man zunächst die
Clausius-Clapeyronsche Gleichung nach $L$ auf.
\begin{equation}
  (V_D - V_F) dp = \frac{L}{T} dT
  \iff L = (V_D - V_F) \frac{dp}{dT} T
\end{equation}
Nun ist der Differentialquotient $\frac{dp}{dT}$ die Ableitung der
Ausgleichsfunktion $f_2(T)$:
\begin{equation}
  f_2'(T) = 200.6 T - 71186.3
\end{equation}
Das Flüssigkeits-Volumen $V_F$ pro Mol wird über die molare Masse und die
Dichte von Wasser\cite{stoffwerte} bei $T = 373 K$ und
$p = 100000 \si{\Pa}$ berechnet:
\begin{align}
  M_W & = \underbrace{2 \cdot 1}_{H_2} + \underbrace{16}_{O}
  \si{\gram\per\mol} & \rho_W & = 950000 \si{\gram\per\cubic\meter}
\end{align}
\begin{equation}
  V_F = \frac{M_W}{\rho_W} = 1.9 \cdot 10^{-5} \si{\cubic\meter\per\mol}
\end{equation}
Das Dampf-Volumen $V_D$ pro Mol kann über die folgende Gleichung genähert
werden:
\begin{equation}
  \Bigl(p + \frac{a}{V²}\Bigr)V = R T
\end{equation} mit $T = 373 \si{\kelvin}$, $p = 100000 \si{\Pa}$ und
$a = 0.9 \si[per-mode=reciprocal]{\joule\cubic\meter\per\mol\squared}$.
Durch umformen mit der \\
pq-Formel erhält man dann einen sinnvollen Wert
\begin{equation}
  V_D = 0.030726 \si{\cubic\meter\per\mol}
\end{equation} für das Dampfvolumen.
Zuletzt setzt man die Werte in die umgeformte
Clausius-Clapeyronsche Gleichung ein und erhält für die Verdampfungswärme in
Abhängigkeit von der Temperatur folgende Funktion:
\begin{equation}
  L(T) = 6.160 \cdot T^2 - 2185.918 \cdot T
\end{equation}

\newpage

\subsection{Diskussion der Ergebnisse}
In Abbildung \ref{fig:f1(T)} kann man erkennen, dass die Messwerte sehr nah
an der durch die Funktion linregress bestimmten Geraden liegen. Der $r$-Wert
(r = 0.9924) bestätigt die Annahme, dass die Verdampfungswärme von Wasser im
Druckbereich $p < 1 \si{\bar}$ annähernd konstant ist. Auffällig ist, dass
die berechnete äußere Verdampfungswärme $L_a$, die zur Volumenausdehnung und
damit zur Druckerhöhung im Gefäß nötig ist, nur $\frac{1}{10}$ mal so groß wie
die innere Verdampfungswärme oder auch latente Wärme
$L_i$, die zur Überwindung der molekularen Anziehungskräfte
benötigt wird, ist.
Das liegt daran, dass $L_i$ die kinetische Energie $E_\text{kin}$ der Moleküle in
der Flüssigkeit so erhöhen muss, dass der Anteil mit maximaler kinetischer
Energie wie in der Maxwellschen Geschwindigkeitsverteilung beschrieben wächst.
Die äußere Verdampfungswärme $L_a$ sorgt lediglich dafür, dass die Moleküle
im Wasserdampf, die deutlich weniger gebunden sind und mehr Platz haben,
zusammengedrückt werden.
\begin{align}
  V & = \text{const} & p & > p_0 \\
\end{align}
In Abbildung \ref{fig:f2(T)} erkennt man, dass Druck und Temperatur ab
$p > 1\si{\bar}$ anders voneinander abhängen und die latente Wärme $L$
dadurch nicht mehr als konstant angesehen werden kann. Ein Ausgleichspolynom
kann diese nicht konstante Abhängigkeit angenähert darstellen.
Für das Polynom $f_2(T)$ wurde trotz der Vorgabe in der Aufgabenstellung
der Grad zwei gewählt, da die zurückgegebenen Werte
bei einem höheren Grad zum Teil sogar eine ganze Größenordnung von den
Literaturwerten abweichen.
Das lässt sich dadurch erklären, dass bei höheren Graden auch kleinere
Messfehler einen viel größeren Einfluss auf das Polynom haben. Polyfit
versucht sozusagen das Polynom genau auf die gegebenen Messwerte zu legen,
wodurch die Funktion keinen Ausgleich mehr darstellt.
Zudem kommt eine Ausgleichsfunktion zweiten Grades für $f_2(T)$
den Messwerten für Druck und Temperatur bereits sehr nahe.

Bei der berechneten Verdampfungswärme $L(T)$ muss man beachten, dass in der
zur Berechnung benutzten Formel Drücke und Volumina als konstant
angenommen wurden, was nur stark genähert zutrifft. Den Graphen von $L$ sieht
man in Abbildung \ref{fig:L}.

\begin{figure}[h]
  \centering
  \includegraphics[height = 8.30 cm]{plot3.pdf}
  \caption{Verdampfungswärme bei $p > 1\si{\bar}$, Graph $L(T)$}
  \label{fig:L}
\end{figure}

Der Graph von L(T) hat bei $\SI{373}{\kelvin}$ etwa denselben Wert, wie die
bei Messung 1 bestimmte kostante Verdampfungswärme $L$ im Bereich von
$\SI{273}{\kelvin}$ bis $\SI{373}{\kelvin}$.
Der Unterschied beträgt etwa
\begin{equation}
  L(373) - L = \SI{42169}{\joule\per\mol} - \SI{36122}{\joule\per\mol} =
  \SI{6047}{\joule\per\mol}
\end{equation}
also beträgt die prozentuale Abweichung
\begin{equation}
  1 - \frac{L}{L(373)} = \SI{14.3}{\percent} .
\end{equation}
Bei dem Vergleich mit einem Literaturwert der
Verdampfungswärme\cite{verdampfungswärme}
bei
$\SI{373}{\kelvin}$ von $L_w = \SI{1}{\joule\per\mol}$ beträgt die absolute
Abweichung
\begin{equation}
  L(373) - L_w = \SI{42169}{\joule\per\mol} - \SI{40644}{\joule\per\mol} =
  \SI{1525}{\joule\per\mol}
\end{equation}
und die prozentuale Abweichung
\begin{equation}
  1 - \frac{L_w}{L(373)} = \SI{3.6}{\percent} .
\end{equation}

\newpage
